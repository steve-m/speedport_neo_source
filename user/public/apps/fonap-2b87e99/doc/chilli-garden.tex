
The so called ``Walled Garden'' is the list of allowed network
resources (websites, etc) that are available to visitors prior to any
authentication. Within \texttt{chilli}, walled garden resources
specify the network protocol, destination IP address, and destination
port to allow. Only the IP address is required as the protocol and
port will default to being ``any''.

\subsection{Using uamallowed}

The first, and most standard, configuration is to use
\texttt{--uamallowed} which can have the following forms (as shown in
\texttt{chilli.conf}):

\begin{alltt}\small
# With protocol, hostname, and port
uamallowed tcp:www.coova.org:80

# With a network and port
uamallowed 192.168.1.0/24:80

# With a single IP address
uamallowed 192.168.1.1

# Together in a list separated by comma
uamallowed 192.168.1.1,www.coova.org,192.168.1.0/24:80
\end{alltt}

\textbf{Note}: Hostnames found in \texttt{uamallowed} options are
resolved during the parsing of the configuration file. All IP
addresses returned for the hostname are added to the walled garden.

\subsection{Using uamdomain}

The \texttt{--uamdomain} option can be used to specify a domain
wildcard, allowing all IP addresses resolved under that domain name. 

\begin{alltt}\small
uamdomain coova.org
uamdomain coova.com
uamdomain coova.net
\end{alltt}

\textbf{DNS-based Dynamic Walled Garden}
\begin{alltt}\small
Client                          Chilli          DNS Server
  | DNS Request (www.coova.org)   |                 | 
  |------------------------------>|                 |
  |                               | (Forwarded)     |   (*) DNS responses are inspected by 
  |                               |---------------->|       chilli. If the hostname matches,
  |                            (*)|<----------------|       the associated IP address(es)
  |<------------------------------|                 |       are dynamically added to the 
  | DNS Response (X.X.X.X)        |                 |       walled garden. 
\end{alltt}

\textbf{Note}: The \texttt{uamdomain} option defines one domain per
option instance, but the option can be repeated for multiple domains.

\textbf{Note}: When using DNS-based dynamic walled garden, it does
require that the visitor uses DNS to resolve the hostnames of
websites. This is usually, however, not an issue.

\subsection{Using uamdomainfile}

Taking the concept of DNS-based walled garden further, the
\texttt{--uamdomainfile} option can be used to define a file
containing a list of regular expressions to be checked in determining
if a DNS domain is allowed or not.

\begin{alltt}\small
uamdomainfile /path/to/regex/file
\end{alltt}

The \texttt{uamdomainfile} option specifies the location of the file
containing the regular expression rules. For example:

{\small%
\begin{verbatim}
!^not-allowed.domain.com$
^(.*\.)?domain.com$
^(.*\.)?google.[a-z]{2,3}$
\end{verbatim}%
}

will add all IP addresses resolved from hostnames under
\texttt{domain.com}, except for \texttt{not-allowed.domain.com} (shown
to demonstrate the special \texttt{!} negation prefix, which must come
before any matching allow rule). To demonstrate a more complex regular
expression, we also will allow anything Google.

\textbf{Note}: This option requires build time option
\texttt{--enable-uamdomainfile}.

\subsection{Using ipwhitelist}

To add compatibility for a feature add by FON Wireless, the
\texttt{--ipwhitelist} option can be used to specify a file containing
a binary list of allowed IP addresses (generated by a different
program).

\begin{alltt}\small
ipwhitelist /path/to/binary/file
\end{alltt}

\textbf{Note}: This option requires build time option
\texttt{--enable-ipwhitelist}.
