% Description of the FONERA L2TP implementation (Chillout)
% This is a TeX source file. Convert to PDF using pdftex as follows:
% pdftex fonucs_test_plan.tex
% run twice generate index in the first pass and include it in
%\input manmac
% METAPOST mlgraphics
%\input epsf
%\input ifpdf.sty
\input supp-pdf
%\def\epsfbox#1{\hbox{\convertMPtoPDF{#1}{1}{1}}}

%\input supp-mis.tex
%\input supp-pdf.tex
% fonts
\font\disclaimerfont=cmr10 at 9truept
\font\footsc=cmcsc10 at 8truept
\font\footbf=cmbx10 at 8truept
\font\footrm=cmr10 at 10truept
\footline={\footsc The FONERA L2TP implementation (Chillout). \hfil\footrm\folio}
\font\bigrm=cmr12 at 14pt
\font\titlefont=cmssdc10 scaled\magstep4
\font\sectfont=cmssdc10 scaled\magstep2
\font\subsectfont=cmssdc10 scaled\magstep1
\font\subsubsectfont=cmssdc10
% TOC macros
\newwrite\toc
\openout\toc=\jobname.toc
\def \tocsection[#1]#2{\line{#2 \dotfill #1}}
\def \tocsubsection[#1]#2{\line{\quad#2 \dotfill #1}}
\def \tocsubsubsection[#1]#2{\line{\qquad#2 \dotfill #1}}
% Section title macros
\def \title #1{\centerline{\titlefont #1}\bigskip}
\def \section#1{\bigskip\par{\leftline{\sectfont#1}}\par\smallskip
	\write\toc{\string\tocsection[\the\pageno]{#1}}}
\def \subsection#1{\bigskip\par{\leftline{\subsectfont#1}}\par\smallskip
	\write\toc{\string\tocsubsection[\the\pageno]{#1}}}
\def \subsubsection#1{\smallskip\leftline{\subsubsectfont#1}
	\write\toc{\string\tocsubsubsection[\the\pageno]{#1}}}
% Document header
\def \header#1.  #2.  #3.  #4. {
\centerline{#1 \tt<#4>}\centerline{#2}\centerline{#3}}
\def\\{\hfil\break}

\outer\def\begindisplay{\obeylines\startdisplay}
{\obeylines\gdef\startdisplay#1
  {\catcode`\^^M=5$$#1\halign\bgroup\indent##\hfil&&\qquad##\hfil\cr}}
\outer\def\enddisplay{\crcr\egroup$$}
\chardef\other=12
\def\ttverbatim{\begingroup \catcode`\\=\other \catcode`\{=\other
  \catcode`\}=\other \catcode`\$=\other \catcode`\&=\other
  \catcode`\#=\other \catcode`\%=\other \catcode`\~=\other
  \catcode`\_=\other \catcode`\^=\other
  \obeyspaces \obeylines \tt}
{\obeyspaces\gdef {\ }} % \obeyspaces now gives \ , not \space
\outer\def\begintt{$$\let\par=\endgraf \ttverbatim \parskip=0pt
  \catcode`\|=0 \rightskip=-5pc \ttfinish}
{\catcode`\|=0 |catcode`|\=\other % | is temporary escape character
  |obeylines % end of line is active
  |gdef|ttfinish#1^^M#2\endtt{#1|vbox{#2}|endgroup$$}}
\catcode`\|=\active
{\obeylines\gdef|{\ttverbatim\spaceskip=\ttglue\let^^M=\ \let|=\endgroup}}

\title{The FONERA L2TP implementation (Chillout)}
\header Pablo Mart\'\i{}n.  Department of Firmware Development.  FON
Technology, Madrid, Spain.  pablo@fon.com.
\begintt $Id$
\endtt
{\disclaimerfont
This material contains information which is proprietary and confidential to
Fon Wireless Ltd. It is not allowed to reproduce, copy, divulge or sell all
or any part thereof without prior written consent of an authorized
representative of Fon Wireless Ltd.}
%\par
%\centerline{\sectfont ChangeLog}
%\item {1.0} 2007-07-24 First draft
%\item {1.1} 2007-08-02 Changes after the Aug 1 meeting with Juan Manuel Munoz
%\item {1.2} 2007-08-08 Changes after the Aug 8 meeting with Juan Manuel Munoz \&
%                  Gonzalo Becares.
%\item {1.3} 2007-08-09 Planning added.
%\item {1.4} 2007-10-10 Documented the implementation and the chillout protocol
%\item {1.5} 2010-10-21 Additions and description of the actual implementation.
%                  Planning and notes not relevant to the description removed
%\bigskip
%\par
\bigskip
\centerline{\sectfont Table of Contents}
\input \jobname.toc
\bigskip
\section{Introduction}

In order to comply with legal requirements in the United Kingdom, about
traceability of end-user communications in ISP's, in the La Fonera router
was implemented a way of redirection of the FON users' traffic between the
router and the internet.

The implemented solution manages tunnels between the Fonera and a 
specialiced network device set up by FON. These tunnels carry the
traffic of a user logged to the public signal in La Fonera. The user
traffic is separated and the record of the user activities is
allowed.

The preferred solution was to use L2TP as tunneling creation and control
protocol, as a lightweight way to establish tunnels. L2TP allows the
tunneling of PPP packets across an intervening network (the "La Fonera"'s
client ISP) in a way that is as transparent as possible. This document
explains the components involved inside the La Fonera, and the IPC mechanism
developed to syncronize all those parts together.
%\par
%\hbox{\convertMPtoPDF{chillout.1}{1}{1}}
%\hbox{\convertMPtoPDF{chillout.2}{1}{1}}
%\par
\section{1. L2TP in ``La Fonera''}
There are few open source solutions for implementing a LAC (L2TP Access
Concentrator) in Linux. The ones researched were:

\item {1.} Roaring Penguin L2TPD
\item {2.} Open L2TPD
\item {3.} Xelerance XL2TPD

The first one is the less tested of the three, and is not maintained anymore.
The second one is the more feature-rich of the three, but had lot of
requirements for our limited platform. The third consists of some patches
over the original L2TPD implementation for Linux, by Marc Spencer. Those
patches where made by Xelerance, the company behind OpenSwan, to support
L2TP with IPSec. Xl2tp was choosen, because:

\item {$\bullet$} It runs entirely on userspace, thus allowing the instalation as
   an "add-on" on La Fonera, without disturbing the actual stable
   system.
\item {$\bullet$} has lightweight requirements
\item {$\bullet$} It seems to be easily modifiable (the number of maintainers for the
   code base over the time account for this)

\section{2. Elements involved in the tunnel creation and maintenance}

The PPP sessions must be created when the user logs in, opening the
ppp session and authenticating it with the radius servers.
This puts a somehow peculiar requirement for a PPP client, in
that it must provide the credentials dinamically, instead of the usual
static setup in wich the ISP's PPP user doesn't change over time.

This was implemented over the original pppd code by modifying it's
authentication code (FIXME: patch file)

The user credentials are retrieved from the FON Captive Portal, i.e.
Coova Chillispot. The PPP session is stablished by the xl2tpd process.
This need thight interaction between the chillispot process
(chilli), the L2TP LAC (xl2tpd), and the PPP daemon (pppd). The agent used
to perform this interaction was called ``chilloutd'', and is a
daemon that communicates with the three elements involved
through UNIX-domain sockets.  

\section{3. Tunnel creation sequence}
% (FIXME: tunnel_creation.dia)

\item {1.} Until a public user is associated to the public SSID, xl2tp
remains dormant. One a user is associated, Chillispot notifies
it to chillout, that stores the pair $mac, ip\_addresss$ in
its memory. In this step the tunnel is not yet stablished

\item {2.} Chillispot shows the captive portal to the user, giving the
users computer an IP address and redirecting her to the
captive portal.
\item {3.} The user authenticates itself in the captive portal, as usual.
Through a redirection chillispot receives a one-time password (OTP)
used to authenticate the session in the radius servers.
\item {4.} Chillispot handles the OTP to chillout, that sends the command
to xl2tpd to stablish the tunnel, and launches a pppd instance to 
establish an authenticated PPP session trough the xl2tp tunnel, using the
user and the OTP. 
The tunnel is not authenticated, the server is configured so
as to permit tunnel stablishment from anyone but shut it down
if an authenticated PPP connection over it is not done in a 
certain period of time, about 5 seconds.
\item {5.} The radius authentication then also made with the same one-time
password, directly from chillispot to the radius servers, as if there
were not tunnel, to maintain compatibility in the radius accounting. This
step is in process of being removed in future implementations.
\item {6.} Once stablished the ppp session, all the traffic from/to
the user (her IP address given in 1), must be redirected
to the ppp device created, using iptables. The iptables
calls are done by chillout, associating the local IP address given
to the user
\item {7.} Chillispot allows the traffic for the user private's IP
until logout or ppp session shutdown
\item {8.} Session can be ended by the user (user logout).  by a timer 
(timeout in chillispot), by the radius server (session time expire), by ppp
or tunnel going down. The first two cases are handled by chillout upon
receiving de DHCP release from chillispot. In the third one the DHCP
release is forced into chillispot for it to return to a clean state.

\section{4. User logout sequence}

\item {1.} The logout can have three sources:
\par \begingroup \narrower
 \item {$\bullet$} The user logouts
 \item {$\bullet$} After a certain time of inactivity, chillispot acknowledges the user
  is away, and performs a logout
 \item {$\bullet$} When the session time expires
\par \endgroup
The three of them are handled by chillispot, and the rest of the system 
will be not aware of the details. The logout details (user, ip address)
are passed to the chillout daemon.

\item {2.} Chillout shut down the ppp session.
\item {3.} If it was the last ppp session for the tunnel (the last user), chillout
tells xl2tp to shut the tunnel down also.
\item {4.} Chillout undo the iptables redirections. 

\section{5. Xl2TP tunnel brought down sequence}

\item {1.} When the tunnel is brought down, xl2tp notifies that to chillout
\item {2.} Chillout keeps a list of the users/tunnels currently in use, and
performs the necessary cleanus to acknowledge those tunnels are not
in use, and undo the redirection (2.4.4) for each of them.
\item {3.} Chillispot is not notified of this logouts. The timeouts will happen
in the server as if chillispot (or la fonera) was down.

\section{6. LNS address queue handling}

xl2tp is provided with at least two lns addresses. This lns addresses will
be retrieved from the chillispot configuration file, updated with
chilli\_readconfig.

xl2tp will try to use the first lns on its list, and then the rest, until
the tunnel is created or the list ends.

This did not require a modificition on xl2tpd, but on it's caller (chilloutd)
that retry connections with different call identifiers

%\section{Adressing, authentication, authorization and accounting at the LNS}

%This subject, and the administration of the LNS itself, is beyond the scope
%of this document. The Networking Department will make their own documentation
%on this.

\section{7. Data dictionary}

\subsection{Static data for the xl2tp daemon (Configuration file)}
 \item {$\bullet$} LNS IP address
 \item {$\bullet$} LNS IP port
 \item {$\bullet$} Configuration: retries, timeout, deadtime.
 \item {$\bullet$} Authentication mechanisms used.

\subsection{Data going from pppd to chillout}
 \item {$\bullet$} PPP device name
 \item {$\bullet$} OTP
 \item {$\bullet$} Private IP given to the user upon association (to perform
   iptables redirections)

\subsection{Data going from xl2tpd to pppd (Lauch)}
 \item {$\bullet$} PPP login (user name)
 \item {$\bullet$} OTP retrieved from chillispot

\subsection{Data going from ppp to chillout}
 \item {$\bullet$} Private IP given to the user
 \item {$\bullet$} PPP device created
 \item {$\bullet$} IP address of the tunnel endpoints
 

\section{8. Implementation}

\subsection{8.1 chilloutd}

chilloutd is a user-space daemon, written in C, using standard POSIX system
calls and C library functions, and using libdaemon to perform and reuse the
usual "daemon" boilerplate. It listen on a UNIX domain socket, created upon
the file /tmp/chillout on start. It source files are:
\begintt
chilloutd.c - the daemon itself  
log.c	    - log subsystem
log.h
poption.c   - handy command line options parser
poption.h
\endtt


The daemon is complemented by a library, that is used by all the components
talking to chilloutd, and by chilloutd itself. Altough it is a library, 
programs are not linked against it. I have found it more practical to
compile it into each modified component, using the openwrt buildroot's
Makefiles.

{\tt {\obeylines {\obeyspaces
chillout.c - Only module
chillout.h - Declaration of structures and exported functions
}}}

Chillout communication is made through clear text lines, LF-ended,
through UNIX domain sockets. Each daemon handles two sockets. One is called
the ``write'' socket and one the ``read'' socket. Altough full bidirectional
communication can be implemented with one socket, I wanted an easy way to
associate question and answer, and have the posibility of block in a read()
while waiting for acknoledge on the "write" socket, and buffer incoming
bytes on the read socket (that will be select()'ed afterwards) from the same
client (that can be sending things using the same channel {\sl before} 
sending acknowledge)

\subsection{8.2 The chillout protocol}

\subsection{8.3 Chillispot modifications}

This describe the changes in the patch file {\tt 400-chillispot-chillout.patch})

Apart from the changes in {\tt cmdline.c}, to retrieve a new parameter
with the path of the UNIX domain socket where chilloutd listens, the main
changes are on {\tt chilli.c} where the main event loop is
Apart from the chilloutd initialization itself (the call to 
{\tt chillout\_init},
in the callback called when a dhcp connection is deleted:
\begintt
    if (!do_not_notify) {
        chillout_write(chillout, CHILLOUT_WAIT_FOR_ANSWER | CHILLOUT_WRITE_SOCK, 
                        "dhcp release %.2X:%.2X:%.2X:%.2X:%.2X:%.2X %s",
                        conn->hismac[0], conn->hismac[1], conn->hismac[2],
                        conn->hismac[3], conn->hismac[4],
                        conn->hismac[5], inet\_ntoa(conn->hisip));
    }
\endtt

In {\tt dhcp.c}, the function pointer type dhcp\_release\_mac changes to accept a parameter
saying wheter to notify chilloutd of the release or not.

And in {\tt redir.c} wich contains the main loop of each process respawned by chillispot per
each redirection (each time the captive portal is shown) where the login and password is
sent to chillispot after a successful login:
\begintt
    if (chillout_write(chillout, CHILLOUT_WRITE_SOCK ,
        "login %s %s %s", inet_ntoa(conn.hisip), conn.username, user_password)) {
         log_err(errno, "Error calling chillout_write");
         in_error = 1;
    }
\endtt

\subsection{8.4 Xl2tpd modifications}

To be done.

\subsection{8.5 pppd modifications}

Each pppd instance registers into chilloutd in main.c
\begintt
    if ((chillout = chillout_init(chilloutsocket, "ppp", NULL)) == NULL)
\endtt

In options.c, two parameters are added to the regular pppd command line:

\begintt
    { "chillout", o_string, chillout_path,
      "Set chillout socket path",
      OPT_PRIO | OPT_PRIV | OPT_STATIC, NULL, MAXPATHLEN },

    { "chilloutmac", o_string, chillout_mac,
      "Set mac address of the client that will be sent to chilloutd",
      OPT_PRIO | OPT_PRIV | OPT_STATIC, NULL, MAXPATHLEN },
\endtt

In {\tt upap.c}, the reason for a failed authentication is sent to
chilloutd, wich forwards it to chillispot:

\begintt
\endtt

In {\tt auth.c} the pppd instances states wheter it is correctly authenticated

\begintt
    if (chillout != NULL) {
       if (chillout_mac[0] == '\0') {
               notice("specify chilloutmac");
       } else {
               chillout_write(chillout, CHILLOUT_WRITE_SOCK, "register success %s %s %d %s", 
                               our_name, ifname, getpid(), chillout_mac);
       }
    } else {
       notice("chillout is NULL");
    }
\endtt 


%\subsection{8.6 Testing environment}
%
%The testing environment will be a Debian 4.0 system, with (at least) one
%instance of QEMU running as the test client.
%
%The fonera programs that will be compiled in the test environment are:
%
%\item {1.} chillispot 1.0-coova.4
%\item {2.} ppp 2.4.3
%\item {3.} xl2tpd 1.1.11
%
%In this testing environment, the FON Captive portal, Radius servers and Cisco
%LNS are used.
%
%\subsubsection{8.6.1 Host operating system and qemu images}
%
%For ease of testing, three identical qemu images, with debian lenny installed
%on them will be used, both for host (wich have the coova-chillispot, xl2tpd and
%pppd installation) and for clients (that run firefox on them and act as the
%wifi clients would)
%
%To setup the clients, make sure you have about 2 GB of free space on the
%partition where the chillout source tree has been decompressed/checked out,
%and that qemu and bridge-utils are installed on your system. Then,
%{\bf as root} 
%run {\tt install\_qemu\_images}. It will download the debian netinst iso,
%create 3 qemu images and install debian on them. The images will be located
%in {\tt qemu/debian\_lenny\_chillout\_server.img},
%{\tt qemu/debian\_lenny\_chillout\_client\_A.img}, 
%{\tt qemu/debian\_lenny\_chillout\_client\_B.img} 
%
%Both the compilation inside the host machine and the setup of the clients
%is made using the script {\tt setup\_qemu\_images}. This launches three qemu
%instances with a common network bridge 

The packages needed to build the fon\_ap in a Debian Lenny system are:
\begintt
build-essential autoconf automake autotools-dev libdjbdns-dev 
libtool automake1.9 libpcap-dev
\endtt
%The .deb once compiled is deployed into the image
%debian\_lenny\_chillout\_server.img

%Later on, and if found necessary, those components will also be hosted
%on the test environment, making it completely standalone. The researched
%tools that can be used for this are:
%
%\item freeradius 1.1.4
%\item l2tpns 2.1.21
%\item apache + PHP for a captive portal implementation

\section{References}

\item {$\bullet$} Layer Two Tunneling Protocol. W. Townsley. Network Working Group (RFC)\\
	{\tt http://tools.ietf.org/rfc/rfc2661.txt}
\item {$\bullet$} Point to Point Protocol. Network Working Group (RFC)\\
	{\tt http://tools.ietf.org/rfc/rfc1661.txt}
\item {$\bullet$} Cisco Systems. Cisco IOS Documentation.\\
	{\tt http://www.cisco.com/en/US/docs/ios/12\_0t/12\_0t1/feature/guide/l2tpT.html}\\
	{\tt http://www.cisco.com/warp/public/cc/pd/iosw/tech/l2pro\_tc.htm}
\item {$\bullet$} Xelerance. Xl2tp website\\
	{\tt http://www.xelerance.com/software/xl2tpd/}
\item {$\bullet$} David Bird and contributors. Chillispot source and documentation. Coova-chillispot website.\\
	{\tt http://www.coova.org}
\item {$\bullet$} l2tpns website \\
	{\tt http://sourceforge.net/projects/l2tpns/}
\bye

