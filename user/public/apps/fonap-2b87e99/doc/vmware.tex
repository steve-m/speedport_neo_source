
\section{Running in VMware}

\begin{itemize}
\item Install Ubuntu Desktop 10.10 from ISO CD-ROM
\item Install additional required packages
\item Setup VMware networking
\item Build and install the fonap package
\item Running the FON gateway 
\end{itemize}

\subsection{Installing Ubuntu from CD-ROM}

\href{http://www.ubuntu.com/desktop/get-ubuntu/download}{Download the
  Ubuntu Desktop 10.10} installation CD-ROM ISO file.

Run VMware player and use the create new wizard to setup a new
appliance from a CD-ROM. Find and use the Ubuntu installation ISO
file. Proceed to install the standard Desktop into VMware. 

See \href{https://help.ubuntu.com/community/VMware}{https://help.ubuntu.com/community/VMware} for more information.

\subsection{Installing additional packages}

Install additional required Ubuntu packages to build fonap. 

\begin{verbatim}
sudo apt-get install build-essential subversion automake libtool

sudo apt-get install libdjbdns-dev libdaemon-dev
\end{verbatim}

The standard install includes a package that has to be removed,
otherwise we will get a configuration file conflict.

\begin{verbatim}
dpkg -r dnsmasq-base
\end{verbatim}

\subsection{Setup VMware networking}

Our setup:

\begin{itemize}
\item The host machine has a physical interface \texttt{eth0} which we
  want to use as the subscriber network,
\item VMware player on the host machine already has \texttt{vmnet8} up
  and running a NAT network.
\end{itemize}

Shut down your VMware player if running. Then edit the virtual
machine's \texttt{.vmx} file. In our case, our virtual machine is
named ``Ubuntu'', so we need to edit the \texttt{Ubuntu.vmx} file
found in the virtual machine's directory. Set the following values,
adding them if not there already, or editing an existing entry. 

{\footnotesize%
\begin{verbatim}
ethernet0.connectionType = "custom"
ethernet0.vnet = "/dev/vmnet8"

ethernet1.present = "TRUE"
ethernet1.connectionType = "custom"
ethernet1.addressType = "generated"
ethernet1.vnet = "/dev/vmnet0"
\end{verbatim}%
}

We have configured the guest \texttt{eth0} (ethernet0) interface to
point to the VMware NAT interface. This will give our guest system a
WAN connection through the host system. The guest system will use DHCP
to configure \texttt{eth0}. We also give the guest system
\texttt{eth1} which points to the VMware bridge interface
\texttt{vmnet0}.

Create a helper script that we can run to restart VMware services:

{\footnotesize%
\begin{verbatim}
#!/bin/bash

# Kill existing bridge 
kill $(ps aux | grep vmnet-bridge | grep -v grep | awk '{print $2}')

# Start up the bridge network including our subscriber network eth0
vmnet-bridge -s 6 -d /var/run/vmnet-bridge-0.pid -n 0 -i eth0

# Bring up vmnet0 (not done per default on our host system)
kill $(cat /var/run/vmnet-netifup-vmnet0.pid)
/usr/bin/vmnet-netifup -s 6 -d /var/run/vmnet-netifup-vmnet0.pid /dev/vmnet0 vmnet0
\end{verbatim}%
}

With everything setup, our host system has the following running processes:

{\footnotesize%
\begin{verbatim}
vmnet-netifup -s 6 -d /var/run/vmnet-netifup-vmnet0.pid /dev/vmnet0 vmnet0
vmnet-netifup -s 6 -d /var/run/vmnet-netifup-vmnet1.pid /dev/vmnet1 vmnet1
vmnet-netifup -s 6 -d /var/run/vmnet-netifup-vmnet8.pid /dev/vmnet8 vmnet8

vmnet-dhcpd -s 6 -cf /etc/vmware/vmnet1/dhcpd/dhcpd.conf \
  -lf /etc/vmware/vmnet1/dhcpd/dhcpd.leases \
  -pf /var/run/vmnet-dhcpd-vmnet1.pid vmnet1

vmnet-dhcpd -s 6 -cf /etc/vmware/vmnet8/dhcpd/dhcpd.conf \
  -lf /etc/vmware/vmnet8/dhcpd/dhcpd.leases \
  -pf /var/run/vmnet-dhcpd-vmnet8.pid vmnet8

vmnet-natd -s 6 -m /etc/vmware/vmnet8/nat.mac -c /etc/vmware/vmnet8/nat/nat.conf

vmnet-bridge -s 6 -d /var/run/vmnet-bridge-0.pid -n 0 -i eth0
\end{verbatim}%
}

And the host system has the following network interfaces (not
including the host system's WAN interface, which isn't important).

{\footnotesize%
\begin{verbatim}
eth0      Link encap:Ethernet  HWaddr 00:1c:23:4f:46:10  
          inet6 addr: fe80::21c:23ff:fe4f:4610/64 Scope:Link
          UP BROADCAST RUNNING MULTICAST  MTU:1500  Metric:1
          RX packets:10939 errors:0 dropped:0 overruns:0 frame:0
          TX packets:8923 errors:0 dropped:0 overruns:0 carrier:0
          collisions:0 txqueuelen:1000 
          RX bytes:1397463 (1.3 MB)  TX bytes:7886491 (7.8 MB)
          Interrupt:17 

vmnet0    Link encap:Ethernet  HWaddr 00:50:56:c0:00:00  
          inet6 addr: fe80::250:56ff:fec0:0/64 Scope:Link
          UP BROADCAST RUNNING MULTICAST  MTU:1500  Metric:1
          RX packets:3021 errors:0 dropped:0 overruns:0 frame:0
          TX packets:51 errors:0 dropped:0 overruns:0 carrier:0
          collisions:0 txqueuelen:1000 
          RX bytes:0 (0.0 B)  TX bytes:0 (0.0 B)

vmnet1    Link encap:Ethernet  HWaddr 00:50:56:c0:00:01  
          inet addr:172.16.187.1  Bcast:172.16.187.255  Mask:255.255.255.0
          inet6 addr: fe80::250:56ff:fec0:1/64 Scope:Link
          UP BROADCAST RUNNING MULTICAST  MTU:1500  Metric:1
          RX packets:0 errors:0 dropped:0 overruns:0 frame:0
          TX packets:389 errors:0 dropped:0 overruns:0 carrier:0
          collisions:0 txqueuelen:1000 
          RX bytes:0 (0.0 B)  TX bytes:0 (0.0 B)

vmnet8    Link encap:Ethernet  HWaddr 00:50:56:c0:00:08  
          inet addr:172.16.32.1  Bcast:172.16.32.255  Mask:255.255.255.0
          inet6 addr: fe80::250:56ff:fec0:8/64 Scope:Link
          UP BROADCAST RUNNING MULTICAST  MTU:1500  Metric:1
          RX packets:383 errors:0 dropped:0 overruns:0 frame:0
          TX packets:338 errors:0 dropped:0 overruns:0 carrier:0
          collisions:0 txqueuelen:1000 
          RX bytes:0 (0.0 B)  TX bytes:0 (0.0 B)
\end{verbatim}%
}

We have our subscriber test machines on the \texttt{eth0} interface.

\subsection{Running the FON gateway}

With your VMware guest system running, be sure to run the helper
script mentioned above - this has to always be ran after starting
VMware player.

Inside the running guest system, we have the following network
interfaces:

{\footnotesize%
\begin{verbatim}
eth0      Link encap:Ethernet  HWaddr 00:0c:29:8a:a1:8f  
          inet addr:172.16.32.130  Bcast:172.16.32.255  Mask:255.255.255.0
          inet6 addr: fe80::20c:29ff:fe8a:a18f/64 Scope:Link
          UP BROADCAST RUNNING MULTICAST  MTU:1500  Metric:1
          RX packets:4131 errors:0 dropped:0 overruns:0 frame:0
          TX packets:3761 errors:0 dropped:0 overruns:0 carrier:0
          collisions:0 txqueuelen:1000 
          RX bytes:1048064 (1.0 MB)  TX bytes:410615 (410.6 KB)
          Interrupt:19 Base address:0x2000 

eth1      Link encap:Ethernet  HWaddr 00:0c:29:8a:a1:99  
          inet6 addr: fe80::20c:29ff:fe8a:a199/64 Scope:Link
          UP BROADCAST RUNNING MULTICAST  MTU:1500  Metric:1
          RX packets:4032 errors:0 dropped:0 overruns:0 frame:0
          TX packets:1860 errors:0 dropped:0 overruns:0 carrier:0
          collisions:0 txqueuelen:1000 
          RX bytes:470777 (470.7 KB)  TX bytes:614872 (614.8 KB)
          Interrupt:19 Base address:0x2080 
\end{verbatim}%
}

CoovaChilli is configured with \texttt{eth1} in \texttt{/etc/default/chilli}:

{\footnotesize%
\begin{verbatim}
START_CHILLI=1
DHCPIF="eth1"
CONFFILE="/etc/chilli.conf"
\end{verbatim}%
}

View the chilli logs:

{\footnotesize%
\begin{verbatim}
$ tail -f /var/log/messages 
Nov 11 00:03:18 ubuntu chillispot[1661]: chilli.c: 2498: New DHCP request from MAC=00-14-5E-10-77-F3
Nov 11 00:03:18 ubuntu chillispot[1661]: chilli.c: 2460: Client MAC=00-14-5E-10-77-F3 assigned IP 192.168.182.11
\end{verbatim}%
}
